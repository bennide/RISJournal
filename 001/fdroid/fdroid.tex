\section*{Fdroid Installationsanleitung}
\hypertarget{fdroid}{}
\label{fdroid}
\NewsAuthor{Horst JENS}

\textbf{In diesem Tutorial wird erklärt wie Sie sich auf einem Android Smartphone oder Tablet das Fdroid Repository für freie (free/libre Open Source) Software installieren, um eine sinnvolle Ergänzung zum Google Play Store zu haben.}

%\begin{center}
%\includegraphics[width=\linewidth]{fdroid/horst2011mitdoppeltux.jpg} \\
%\footnotesize{Horst JENS. Bildrechte: [9], cc-by-sa}
%\end{center}
\begin{center}
\includegraphics[width=\linewidth]{fdroid/fdroid2.png} \\
\footnotesize{Fdroid}
\end{center}
\subsection*{Das freies Softwarerepository Fdroid für Android installieren}

Wirklich \href{http://de.wikipedia.org/wiki/Freie_Software}{\textit{freie (free/libre Open Source) Software [1]}} für \href{http://www.android.com/}{\textit{Android [2]}}-Geräte zu finden ist gar nicht so leicht: Der \href{https://play.google.com/store}{\textit{Google Play Store [3]}} lässt sich zwar nach \textit{kostenlos} durchsuchen, aber kostenlos (\textit{free as beer}) ist eben nicht das selbe wie wirklich frei (quelloffen, \textit{free as freedom}) im Sinne der \href{http://de.wikipedia.org/wiki/GNU_General_Public_License}{\textit{GPL Lizenz [4]}}. Mit relativ wenig Aufwand ist es allerdings möglich, auf Android-Geräten (\textit{Devices} wie Smartphones, Tablets, ...) ein alternatives Softwareverzeichnis, ein sogenanntes \href{http://de.wikipedia.org/wiki/Repository}{\textit{Repository [5]}} zu installieren, welches nur wirklich freie Software anbietet. Keine \href{http://de.wikipedia.org/wiki/Crippleware}{\textit{Crippleware [6]}}, keine nervenden Werbebanner, kein geheimer Programmcode. Dieses Tutorial erklärt wie es geht:


\subsection*{Anleitung:}
Die folgende Schritte möglichst direkt mit dem Android-Gerät ausführen:

\textbf{Gerätezugriff gewähren}: Die FDroid-\textbf{App} verlangt beim Installieren weitgehende Rechte auf dem Android-Gerät, um weitere Apps installieren und löschen zu dürfen:

\begin{itemize}
\item Im Systemmenü des Android-Geräts unter \emph{Einstellungen - Sicherheit} die Option \emph{Unbekannte Herkunft - Installation von Apps aus anderen Quellen als dem PlayStore zulassen} anklicken (das Häckchen muss gesetzt sein).
\item Die Website \url{http://f-droid.org/} ansurfen und die Seite durchlesen. (Die Seite akzeptiert Spenden per Flattr, PayPal und Bitcoin.) Im Fdroid FAQ (im Wiki) wird u.a. erklärt, warum FDroid NICHT im Google Play Store drin ist (Konkurrenzverbot).
\item Direkt auf der FDroid-Website kann man mit dem Menüpunkt \emph{Browse} schauen was für freie Apps es gibt, um zu entscheiden ob man FDroid überhaupt braucht.
\item Direkt auf der \href{http://f-droid.org/}{\textit{FDroid-Website [7]}} ist ein Download-Link mit dem man die Datei \href{http://f-droid.org/FDroid.apk}{\textit{FDroid.apk [8]}} downloaden kann. Oder man scannt mit dem Android-Device diesen QR-Code:

\begin{center}
\includegraphics[width=4cm]{fdroid/fdroidurl.png}
\end{center}

\item Die heruntergeladene Datei \emph{FDroid.apk} öffnen. Dazu oben in der Statusleiste auf das Downloadsymbol (Pfeil nach unten) mit dem Finger wischen und auf die Meldung \emph{Download abgeschlossen} drauf klicken. Nach einigem Nachfragen wird die FDroid-App auf dem Android-Device installiert.
\item Die FDroid-App im App-Menü des Android-Geräts suchen und starten.
\item Wenn die FDroid-App läuft, den Menü-Button drücken und \emph{Update} wählen. Nach einiger Zeit erscheint eine Liste verfügbarer freier Apps, die man mittels \textit{Install}, \textit{Update} und \textit{Remove} managen kann.
\item Um zu schauen ob es neuere Versionen von freien Apps gibt muss man FDRoid starten und im Menü auf \textit{update} klicken. Die fdroid-Apps werden NICHT über den Google Play Store aktualisiert.
\item Es empfiehlt sich manchmal, nicht die allerneuste Version einer App zu installieren sondern die vorletzte Version (die 2. von oben in der Versionliste). Dadurch verzichtet man zwar auf die allerneusten Features, erspart sich dafür die allerneusten Fehler.
\item FDroid selbst lässt sich mittels FDroid updaten :-)
\end{itemize}

\subsection*{Fachbegriffe:}

~~~\href{http://de.wikipedia.org/wiki/Freie_Software}{\textbf{freie Software [1]}}: ist nicht zu verwechseln mit kostenloser Software (Freeware). Freie Software gewährt die \textbf{4 Grundfreiheiten} gemäß der \textbf{GPL-Lizenz} und ermöglicht es, den Quellcode (SourceCode) der Software anzuschauen, zu benutzen, zu verändern und zu kopieren und weiterzugeben.

\href{http://de.wikipedia.org/wiki/GNU_General_Public_License}{Gnu Public License (GPL)} [4]
\begin{center}
\includegraphics[width=4cm]{fdroid/fdroid_gpl_logo.png}
\end{center}
garantiert dem Nutzer der Software die 4 Grundfreiheiten \textbf{use, study, share, improve}. Näheres dazu können Sie im Leitartikel dieser Ausgabe nachlesen. \\

\href{http://www.android.com/}{\textbf{Android}} [2]: Ein freies, Linux-basiertes Betriebssystem, vor allem für Smartphones und Tablets. Android selbst ist GPL-lizensierte, freie Software, aber viele der Apps die auf Android laufen (z.B. Skype, aber auch der \textbf{Google Play Store} sind unfreie Software.. \\

\href{http://de.wikipedia.org/wiki/Google_Play}{\textbf{Google Play Store}} ist ein von Google vorinstallierter Softwareshop für Android-Geräte, ähnlich Apple's \href{http://de.wikipedia.org/wiki/AppStore}{AppStore}. Derzeit (Dez 2013) lässt sich Google's Play Store zwar nach kostenlosen Apps filtern, nicht aber nach freien Lizenzen. \\

\href{http://de.wikipedia.org/wiki/Repository}{\textbf{Software-Repository}} [5]: Ein Verzeichnis installierbarer Software (oder Apps) welches eine Versionsverwaltung beinhaltet und das installieren, löschen, updaten und verwalten von Software oder Apps erlaubt. Im Linux-Bereich hat fast jede Distribution ein eigenes Repository (z.B. Debian, Ubuntu, Suse...). Man kann sich auch selbst Repositorys erstellen. \\

\href{http://de.wikipedia.org/wiki/Crippleware}{\textbf{Crippleware, Krüppelware}}: ein verächtlicher Begriff für kostenlose Versionen von Software, deren Leistungsumfang stark eingeschränkt ist und erst nach einer entsprechenden Zahlung sinnvoll verwendbar ist. Viele Apps im Google Play Store sind in der "kostenlos" Variante als Crippleware zu bezeichnen, da ihnen entweder wichtige Features fehlen oder weil sie den Nutzer ständig durch Werbebanner nerven, bis er ein kostenpflichtiges Upgrade zur Vollversion durchführt.

\textbf{Danksagung:} \\
Die Android-Screenshots wurden von Phillip Seibt erstellt.

\subsection*{Download, Feedback:}
\textbf{R.I.S.-Journal}, Ausgabe 001: \\
\href{http://spielend-programmieren.at/de:ris:001}{spielend-programmieren.at/de:ris:001}\\
\textbf{Download} Ordner, verschiedene Formate: \href{http://spielend-programmieren.at/risjournal/001/fdroid}{\texttt{spielend-programmieren.at/\\risjournal/001/fdroid}} \\
\textbf{Feedback} \Letter\ \texttt{horst.jens@spielend-\\programmieren.at} \\



\subsection*{Lizenz, Quellen:}
\begin{wrapfigure}{l}{2.0cm}
\includegraphics[width=2cm]{fdroid/ccbysa88x31.png}
\end{wrapfigure}
Dieses Material steht unter der Creative-Commons-Lizenz Namensnennung - Weitergabe unter gleichen Bedingungen 4.0 International. Um eine Kopie dieser Lizenz zu sehen, besuchen Sie \url{http://creativecommons.org/licenses/by-sa/4.0/deed.de}. \\

\textbf{Quellen:} \\
{[}1{]} \href{http://de.wikipedia.org/wiki/Freie_Software}{wikipedia/Freie-Software} \\
{[}2{]} \href{http://www.android.com}{android.com} \\
{[}3{]} \href{https://play.google.com/store}{play.google.com} \\
{[}4{]} \href{http://de.wikipedia.org/wiki/GNU_General_Public_License}{http://goo.gl/4UTW5B} \\
{[}5{]} \href{http://de.wikipedia.org/wiki/Repository}{de.wikipedia.org/wiki/Repository} \\
{[}6{]} \href{http://de.wikipedia.org/wiki/Crippleware}{de.wikipedia.org/wiki/Crippleware} \\
{[}7{]} \href{http://f-droid.org/}{f-droid.org} \\
{[}8{]} \href{http://f-droid.org/FDroid.apk}{f-droid.org/FDroid.apk} \\
{[}9{]} \href{http://spielend-programmieren.at}{spielend-programmieren} 
