
H: Wer bist du, was machst du gerade (Ausbildung) und wie alt bist du?


D: Hallo, mein Name ist Daniel Rittmannsberger und ich bin 15 Jahre alt. Im Moment besuche ich die htl donaustadt im 22. Bezirk.





H: Wie bist du zur FSFE und zu freier Software gekommen?


D: Das war als ich auf der GameCity war und zu deinem Stand ging, da ich eine Visistenkarte wollte, traf ich den FSFE-Aktivisten
   Franz Gratzer, da dieser am selben Stand die FSFE anpries. Er erklärte mir sofort voller Elan alles über freie Software und so
   unterhielten wir uns den restlichen Tag über freie Software. Bevor ich mit ihm gesprochen hab, hatte ich mich noch nie
   richtig dafür Interessiert aber er hatte so viel Motivaton, dass er es schaffte mich dafür zu begeistern. Das restliche
   Wochenende "Arbeitete" ich noch am Stand, da es mir sehr viel Spaß machte. Danach ging ich zu den FSFE Treffen und
   informierte mich mehr über freie Software.





H: Kennst du viele Jugendliche Deines Alters mit ähnlichen Interessen ?


D: Nein, niemanden und in meiner Klasse ist es auch schwer Leute für freie Software zu begeistern, da sie nur Videospiele
   spielen wollen und die gehen meistens nur auf Windows.





H: Bechreibe ein wenig Deinen Schultyp (Gymnasium / HTL). Inwiefern bist Du in Deiner Schule / seitens Deiner Lehrer 
   mit freier Software in Berührung gekommen? Was habt Ihr darüber gelernt? Was hast du dir selber beigebracht?


D: Im Gymasium (mit 10 Jahren) hatten wir auf den Schulrechnern Ubuntu und Libre Office installiert. Jedoch wurde uns nicht
   erklärt, dass es freie Software ist.
   In der HTL (Höhere Technische Lehranstalt) welche ich momentan im ersten Jahr besuche (10. Klasse), haben wir bisher nur
   einmal kurz darüber gesprochen was freie Software ist und uns Ubuntu angeschaut. Ansonsten haben wir nichts darüber gemacht und auf
   allen Rechnern läuft Windows mit Microsoft Office. Deshalb musste ich mir selber beibringen was freie Software ist, obwohl es nicht
   mehr wirklcih nötig war da ich auf der GameCity sehr viel erfahen hab.





H: Wie war Dein erster Eindruck von FSFE ?


D: Sehr gut! Beim ersten Treffen auf dem ich war wurde ich sehr nett aufgenommen und es wurde über viele interessante
   Dinge gesprochen. Ich aknn mir jedoch vorstellen, dass es für Leute die nicht sehr interessiert sind langweilig sein kann,
   da sehr viel über Fachliches gesprochen wurde. Trotzdem konnte ich fast immer folgen und verstand alles. Mein erster
   Eindruck war also sehr positiv.





H: Wie wirkten die FSFE Aktivisten auf dich? Welche Nerd-Klischees (sofern du welche hattest) wurden bestätigt, 
   welche nicht?


D: Sie wirkten sehr nett und motiviert. Ich habe mir früher unter Nerds immer stark übergewichtige Männer mit Brille und Glatze
   vorgestellt. Aber da ich jetzt selber ein Nerd bin sind diese Klischees zerstört und selbst wenn ich diese Klischees noch
   hätte wäre mir dort niemand als richtiger Nerd vorgekommen. Alles in allem wirkten sie sehr motiviert, lustig und nett.





H: Kamst du dir seltsam / gelangweilt vor als einziger Jugendlicher?


D: Nein, überhaupt nicht. Da ich sehr nett empfangen wurde kam ich mir nicht seltsam vor, obwohl dort jeder mindestens 20 war.
   Und gelangweilt hat es mich kein bisschen da ich mich für jedes Thema begeistern konnte und interessiert zugehört habe.





H: Wie würdest du die Ziele von FSFE beschreiben?


D: Meiner Meinung nach sind die Ziele der FSFE auf möglichst vielen Rechnern irgendwelche Linux Distributionen zu installieren
   da Windows für alles Geld verlangt und man nichts daran ändern kann.





H: Was war dein seltsamstes / coolstes / lustigstes Erlebnis im Zusammenhang mit FSFE bisher?


D: Die GameCity, die Treffen jeden dritten Freitag im Monat und dass ich sofort nachdem ich bei der Game City geholfen hab, die
   Möglichkeit bekam, gratis ein Unterstützer zu werden.





H: Eine Message für junge Leser ?


D: Schaut euch auf jeden Fall Linux Distributionen und allgemein freie Software an, denn sie kann genau dasselbe wie
   properitäre Software kostet aber nicht und man kann mehr damit machen. Und falls euch freie Software interessiert kommt
   zu den Treffen der FSFE. In Wien finden die jeden dritten Freitag im Monat im Metalab (Rathausstraße 6) statt.




