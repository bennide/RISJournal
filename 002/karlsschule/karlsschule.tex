Ich war seit ich mich erinnern kann immer schon ein Windows-Hasser, aber -Verwender. Ich habe immer das Gefühl gehabt das ist nix gutes was ich da verwende, aber ich habs halt gemacht und gleichzeitig war da diese gewisse Angst das man Illegales in einer Schule verwendet. Wir sind eine kleine Hauptschule (neue Mittelschule) mit ca. 200 Schülern. Wir sind eine evangelische Schule, es gibt kaum noch evangelische Schüler, die Kirche hat auch kein richtiges Interesser daran, wir sind übernommen worden von der Diakonie (eine kirchliche Vorfeldorganisation, ähnlich der katholischen Caritas, die sich eigentlich mit Flüchtlingen usw beschäftigt). Die Diakonie hat gesagt sie übernimmt uns als Schule auch noch weil sie uns sinnvoll findet usw. 

...

Wir sind von der Diakonie übernommen worden und der Diakonie-Informatiker,  Erich S., ist zu mir gekommen und ich habe ihm gesagt wie es damals ausschaut, "eigentlich ist das alles was wir da haben illegal (windows, office usw), was können wir da machen?" er sagt "naja,  illegale Sachen mache ich nicht, warum verwendet ihr als Schule nicht Linux?"  Ich antwortete: "ich bin voll dafür, aber ich kenn mich nicht aus". Erich hat gesagt er macht das. 

Rückblickend betrachtet war das eine Zeit wo er versucht hat in der Diakonie Linux durchzusetzen (ca. 2006, über 400 Computer, noch ohne Schulen). Er hat mir Ubuntu gezeigt, ich war voll begeistert und habe mir mir einen 2. Computer damit aufgesetzt. Nach ca. einem Jahr hat Erich mich gefragt was ich zu Hause als "Hauptcomputer" verwende. Ich sagte Windows und er meinte, solange ich nicht den Hauptcomputer auf Linux umstelle werde ich Linux nie lernen. Ich hab mir gedacht "naja, ganz schön hart" aber ich habe mir was sagen lassen und habe dann ein Dual-Boot eingerichtet. Das war ein Riesen Schritt für mich das ich gemerkt habe "Linux ist mein Operating System" und nicht "etwas zum Ausprobieren". In der Schule war es auch so, am Anfang haben wir 2 Geräte gehabt und dann immer mehr und nach einem Jahr haben wir wirklich umgestellt.  Das traurige ist dass es nicht gut ausgegangen ist. 2007 nur noch Ubuntu, mit der vorhandenen Hardware. Einmal haben wir günstig Laptopos angeschafft. Bis 2013 mit Linux betrieben.

Meine Erfahrung war die.: Es hat ein paar erbitterte Gegner gegeben, ca. 3 Kollegen von insgesamt 30. Kollegen mit denen ich sonst gut ausgekommen bin, aber wo ich gemerkt habe das war keine kleine Aversion, das war erbitterte Feindschaft. Das waren zum Teil Leute die sich schon viel beschäftigt haben mit Computern, nicht absolte Anfänger. Z.B. waren so Kleinigkeiten, der eine Kollege wollte mit Excel was eine Potenzrechnung machen. In (Libre-Office Calc) geht das an sich genau so, nur gab es ein Problem mit dem Hoch-Zeichen ( ^ Dead Key und Leerzeichen ) Im Calc geht das ein bissl anders, ich bin erst ein Jahr später draufgekommen wie, und der Kollege hat das Gefühl gehabt er kann mit dem neuen System nicht arbeiten.  

Hätten wir schon vorher (vor der Umstellung auf Linux) OpenOffice verwendet wäre das kein Problem gewesen. In unsere Schule war es relativ harmlos. 3 Gegner, dann ein Direktorenwechsel, der neue Direktor war sehr offen für die Linuxkritiker.

Von den Eltern gab es nie eine einzige Beschwerde

Von den Schülern, manche sind richtig drauf aufgesprungen wenn ich ihnen gezeigt habe schau da kannst selber etwas installieren,  die haben mich links und rechts überholt. Nur sind die Schüler immer wieder auf Windows abgefahren, wegen Spielen ?

Für manche Schüler war das so normal das bei uns Linux installiert ist, die haben sich erst aufgeregt als wieder Windows da war. Da habe ich dann erst festgestellt dass die Schüler auch zu Hause Linux verwenden. Positive Nebeneffekte. Ich finde Linux ist das Einzige (Operation System) dass man an einer Schule verwenden sollte, weil man eben "machen kann", weil man keine juristischen Einschränkungen hat. 

Meine Meinung ist das wenn es nur um unsere Schule gegangen wäre hätten wir weiterhin Linux. Parallel zu uns gab es aber auch Vorgänge in der Diakonie, die sich auf uns auswirkten: Dort hat es einen Sturm der Entrüstung der Diakonie - Mitarbeiter gegeben (Word + Excel-User) die dachten die Welt geht unter. Deshalb hat die Diakonie auf Windows zurück umgestellt um sich nicht das Geraunze der Mitarbeiter anhören zu müssen. 

Kostenfrage: Bei mir ging es beim Linux-Verwenden um Freiheit vs. Kontrolle durch einen Konzern. Ich und der Erich haben versucht in der Diakonie Schulungen zu machen und Leute von diesem Gedanken der Freiheit zu überzeugen, was uns letztlich aber nie gelungen ist. Das sind Leute die kaufen Fair-Trade Kaffee, auch wenn er vielleicht nicht ganz so gut schmeckt, das ist selbstverständlich das man trotzdem Fair Trade Kaffee kauft, weil das eine gute Sache ist. Von den Diakonie Mitarbeitern / Lehrern gab es vielleicht 5 Leute die auf das Thema überhaupt eingestiegen sind (Warum Linux), allen anderen ist das Thema Computer entweder egal oder sogar zuwieder, die möchten sich damit nicht beschäftigen. Das ist für die so wie wenn man einen Fernseher einschaltet. Dann gibt es die (Lollegen) die Angst gehabt haben dass sie jenes Wissen welches sie sich erlernt haben nicht mehr anwenden können - da ist mir leider nicht gelungen dagegen zu wirken.

Dann kam noch das Argument "Wir brauchen das Beste für unsere Schüler, wir sollen nicht sparen in diesem Bereich". Darauf habe ich gesagt es geht nicht um's sparen, die Kostenfrage (Linux kostet nichts) ist nicht das Hauptargument für einen Umstieg auf Linux, es geht darum dass Linux das Beste  ist um die Kinder zu unterrichten.

Ich verwende ja für den Geographieunterricht keinen Neckermann-Katalog sondern einen Atlas. 

Wie dann die 2013 Rück-Umstellung auf Windows kam gab es -erstaunlich für mich- ein paar Gegenstimmen die für Linux waren, es gibt auch Kollegen die sich daheim Linux installiert haben und es privat weiterhin verwenden. 

Das Problem war die Trägerorganisation (Diakonie) wo wir als Schule der einzige Bereich mit Linux waren. Außerdem gab es das Problem beim Support, der Erich hat immer sehr gute Zivildiener gehabt mit guten Linuxkenntnissen. Die haben natürlich nach dem Zivildienst gute Jobangebote bekommen und sind nicht lange bei der Diakonie geblieben, aber es war zumindest immer jemand da der sich gut ausgekannt hat. 

Als Erich bei der Diakonie kündigte kümmerte sich niemand mehr um die Linux-Qualifikation des IT-Supports, da merkte ich auf einmal dass ich mehr Ahnung habe von Linux als der IT Support.  

Meine Analyse, was hat Erich falsch gemacht. Ich denke er hat es im Prinzip gut gemacht.  Der größte Fehler war zu wenig auf die User einzugehen. Man hätte es viel langsamer machen müssen, den Leuten die Umstellung "verkaufen", sodass sie von selber fodern "wann bekommen wir endlich das neue Linux System", dann hätte es wahrscheinlich besser funktioniert.

Ich denke bei einer Schule ist das wichtigste dass man den Kindern bewusst macht und hofft das was hängen bleibt. Der "Wunsch nach Freiheit", das ist für mich so ein bissl wie der Wunsch nach Demokratie in einer Monarchie. Die Leute wissen zwar das es etwas Besseres gibt aber die meisten sagen sich "Jaja, der Kaiser, passt eigentlich eh". trotzdem gibt es Leute die wollen mehr, die sich denken "Ist mir wurst ob der Kaiser gerade in Ordnung ist oder nicht, ich will das selber in die Hand nehmen" - genau das ist Open Source. 

Ich habe immer wieder versucht das meinen Kollegen klar zu machen aber est halt ..tja. 

Ich erzähle den Kindern immer warum ich open source verwende und dass es einen Unterscheid gibt zwischen gratis (versteckte Kosten) und frei.

Aber es ist letztendlich was fremdes in usnerer Kultur dass man etwas macht um der Sache willen, z.B. Wanderwege markieren damit man besser wandern kann. Die Kinder können sich auch schwer vorstellen dass man damit Geld verdienen kann.

Einige ganz wenige machen sich darüber Gedanken dass sie bei Open Source selbst was machen können, aber den meisten ist das egal.

Open Source Förderung durch Forken/Sharen von Scratch Projekten

Ich sehs trotzdem ein bissl positiv, wir als Schule verwenden nach wie vor freie Software, halt auf Windows (Gimp, Inkscape) und ich versuche den Kindern klar zu machen dass Open Source ein Wert ist. 

Ich glaube man müsste bei den Schulen anfangen, damit man Leute hat die mit freien Betriebssystemen umgehen können, die dann nicht aufschreiben wenn Sie kein Microsoft Word haben.
