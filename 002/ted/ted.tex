Wie ich als Lehrer zum Film kam
von Jörg Bilert bilert@gmail.com
Jörg Bilert ist Lehrer am Georg-Wilhelm-Steller Gymnasium in Bad Windsheim (Mittelfranken/ Bayern/ Deutschland). Er unterrichtet Englisch und evangelische Religion, betreut einen Wahlkurs zum Thema LEGO Mindstorms und ist Systembetreuer und Website-Administrator). Seine Interessen gelten Bildungsfragen, Opensource, aber auch Videospielen und Hardwarehacks (obwohl er zwei linke Hände hat). Er freut sich über Anregungen und Kommentare auf Twitter (@talmargrosskotz).


Zugegeben, die Überschrift verspricht mehr als sie halten kann. Ich war weder in Hollywood, noch in den Bavaria FIlmstudios. Eigentlich habe ich nur die Untertitel zu einem kurzen Film geschrieben. Alles begann vielmehr mit einem kurzen Artikel in der Zeitschrift c’t, in dem die TED-Talks vorgestellt wurden. Die Idee von TED (steht übrigens für: Technology- Entertainment-Design) hat mich sofort fasziniert. Eine frei zugängliche Plattform, auf der Menschen aus den verschiedensten Fachbereichen und Hintergründen ihre Erfahrungen, Entdeckungen und Entwicklungen mit der Welt teilen können. Ein Mekka für jeden vielseitig interessierten und weltoffenen Menschen wie mich. Auf der Homepage treffen Gedanken von Al Gore auf die Überlegungen von Schülern, oder folgen die neusten Erkenntnisse der Bionik auf eine Zaubershow zum Thema Wahrnehmung.


Nachdem TED für einige Zeit das Fernsehprogramm in meinem Haus komplett ersetzte, stellte ich fest, dass man sich an diesem Projekt beteiligen kann. Man muss aber dazu nicht selbst einen Vortrag halten, sondern kann auch helfen, indem man den Vortrag transkribiert oder mit Untertiteln versieht. Die TED-Community sucht auch immer nach sprachlich begabten Leuten, die die Untertitel und Transkripte in andere Sprachen übertragen.


Damit war ich als Englischlehrer herausgefordert und ich nahm die Herausforderung an. Ich registrierte mich also als Übersetzer bei TED. Dazu muss man seine Qualifikation und Motivation angeben, was gut ist denn gerade bei solchen Vorträgen sollte die Qualität stimmen. Anschließend dürfte ich mir einen Talk aussuchen für den ich die deutschen Untertitel schreiben wollte. Meine Wahl fiel auf: Noah Feldman:Politics and religion are technologies. 
Nun hatte ich 30 Tage Zeit meine Arbeit bei TED einzureichen. Was ich zunächst für ein Kinderspiel hielt, stellte sich bald als sehr ernst zu nehmende Herausforderung dar. Zunächst war das Thema sprachlich sehr komplex und mit Begriffen aus Soziologie, Politik und Theologie gespickt. Zweitens musste die Übersetzung nicht nur möglichst genau sein sondern auch dem Fluss des Vortrags folgen und zu den Englischen Untertiteln passen. Da Herr Feldmann sehr frei redete, war es nicht immer einfach die Übersetzung kohärent zu halten. Was ich zunächst für eine reine sprachliche Fingerübung hielt, hielt mich nun gehörig auf Trab und ich benötigte tatsächlich fast den kompletten Monat (bei ca. 1 Stunde pro Tag) um 15 Minuten Vortrag so zu übersetzen, dass ich zufrieden war. 
Entsprechend erleichtert war ich, als meine Übersetzung die Überprüfung durch Alex Boos überstand. Jede Übersetzung wird durch ein anderes Mitglied kontrolliert. Bei Fehlern, Unstimmigkeiten oder sonstigen Problemen sind der Übersetzer und der Prüfer aufgefordert sich zu einigen bevor die Übersetzung online geht.
Und so geschah es, dass ich .. naja.. eigentlich meine Übersetzung in einem Film zu sehen ist, wenn man den Vortrag von Noah Feldmann mit deutschen Untertiteln schaut oder sich das Transkript durchliest. 
Seit damals hat TED übrigens das Tool für die Untertitel gewechselt. Das Werkzeug der Firma dotSUB wurde durch die Opensource Software Amara ersetzt, mit der ich noch keine Erfahrungen gesammelt habe. Aber wenn ich so darüber nachdenke, juckt es mich eigentlich schon wieder in den Fingern. Ich bin dann mal auf ted.com.


